\documentclass{article}

\usepackage{hyperref}
\usepackage{xspace}
\usepackage[usenames,dvipsnames]{color}
\usepackage[margin=1in]{geometry}
\hypersetup{colorlinks=true}
%\usepackage{ucs}
\usepackage[utf8x]{inputenc}
\begin{document} 
\hypertarget{tex_codes_for_various_unicode_32_characters}{}\subsection*{{TeX codes for various Unicode 3.2 characters}}\label{tex_codes_for_various_unicode_32_characters}

Markus Kuhn -- 2002-07-25

This UTF-8 example file lists the various characters supported by Donald Knuth's TeX, as listed in Appendix F and section 9 of the TeXbook and section 3.3.2 of Leslie Lamport's \LaTeX\xspace  User's Guide and Reference Manual.

Math mode symbols:

Lowercase Greek letters

\begin{verbatim}α \alpha	ι \iota		ϱ \varrho
β \beta		κ \kappa	σ \sigma
γ \gamma	λ \lambda	ς \varsigma
δ \delta	μ \mu		τ \tau
ϵ \epsilon	ν \nu		υ \upsilon
ε \varepsilon	ξ \xi		ϕ \phi
ζ \zeta		ο o		φ \varphi
η \eta		π \pi 		χ \chi
θ \theta	ϖ \varpi	ψ \psi
ϑ \vartheta	ρ \rho		ω \omega
\end{verbatim}
Uppercase Greek letters

\begin{verbatim}Γ \Gamma	Ξ \Xi		Φ \Phi
Δ \Delta	Π \Pi		Ψ \Psi
Θ \Theta	Σ \Sigma	Ω \Omega
Λ \Lambda	Υ \Upsilon
\end{verbatim}
Miscellaneous symbols

\begin{verbatim}ℵ \aleph	′ \prime !	∀ \forall
ℏ \hbar		∅ \emptyset	∃ \exists
ı \imath	∇ \nabla	¬ \neg
j \jmath !	√ \surd		♭ \flat
ℓ \ell		⊤ \top		♮ \natural
℘ \wp		⊥ \bot		♯ \sharp
ℜ \Re		∥ \|		♣ \clubsuit
ℑ \Im		∠ \angle	♢ \diamondsuit
∂ \partial	△ \triangle	♡ \heartsuit
∞ \infty	\ \backslash	♠ \spadesuit
□ \Box		◇ \Diamond
\end{verbatim}
“Large” operators

\begin{verbatim}∑ \sum		⋂ \bigcap	⨀ \bigodot
∏ \prod		⋃ \bigcup	⨂ \bigotimes
∐ \coprod	⨆ \bigsqcup	⨁ \bigoplus
∫ \int		⋁ \bigvee	⨄ \biguplus
∮ \oint		⋀ \bigwedge
\end{verbatim}
Binary operations

\begin{verbatim}± \pm		∩ \cap		   ∨ \vee
∓ \mp		∪ \cup		   ∧ \wedge
∖ \setminus	⊎ \uplus	   ⊕ \oplus
⋅ \cdot		⊓ \sqcap	   ⊖ \ominus
× \times	⊔ \sqcup	   ⊗ \otimes
∗ \ast		◁ \triangleleft	   ⊘ \oslash
⋆ \star		▷ \triangleright   ⊙ \odot
⋄ \diamond	≀ \wr		   † \dagger
∘ \circ		◯ \bigcirc	   ‡ \ddagger
∙ \bullet	△ \bigtriangleup   ⨿ \amalg
÷ \div		▽ \bigtriangledown ⊴ \unlhd
⊲ \lhd		⊳ \rhd		   ⊵ \unrhd
\end{verbatim}
Relations

\begin{verbatim}≤ \leq		≥ \geq		≡ \equiv
≺ \prec		≻ \succ		∼ \sim
≼ \preceq	≽ \succeq	≃ \simeq
≪ \ll		≫ \gg		≍ \asymp
⊂ \subset	⊃ \supset	≈ \approx
⊆ \subseteq	⊇ \supseteq	≅ \cong
⊑ \sqsubseteq	⊒ \sqsupseteq	⋈ \bowtie
∈ \in		∋ \ni		∝ \propto
⊢ \vdash	⊣ \dashv	⊨ \models
⌣ \smile	∣ \mid		≐ \doteq
⌢ \frown	∥ \parallel	⊥ \perp
⊏ \sqsubset	⊐ \sqsupset	⨝ \Join
\end{verbatim}
Negated relations

\begin{verbatim}≮ \not<		  ≯ \not>	    ≠ \not=
≰ \not\leq	  ≱ \not\geq	    ≢ \not\equiv
⊀ \not\prec	  ⊁ \not\succ	    ≁ \not\sim
⋠ \not\preceq	  ⋡ \not\succeq	    ≄ \not\simeq
⊄ \not\subset	  ⊅ \not\supset	    ≉ \not\approx
⊈ \not\subseteq	  ⊉ \not\supseteq   ≇ \not\cong
⋢ \not\sqsubseteq ⋣ \not\sqsupseteq ≭ \not\asymp
\end{verbatim}
Arrows

\begin{verbatim}← \leftarrow	      ⟵  \longleftarrow		↑ \uparrow
⇐ \Leftarrow	      ⟸  \Longleftarrow		⇑ \Uparrow
→ \rightarrow	      ⟶  \longrightarrow	↓ \downarrow
⇒ \Rightarrow	      ⟹  \Longrightarrow	⇓ \Downarrow
↔ \leftrightarrow     ⟷  \longleftrightarrow	↕ \updownarrow
⇔ \Leftrightarrow     ⟺  \Longleftrightarrow	⇕ \Updownarrow
↦ \mapsto	      ⟼  \longmapsto		↗ \nearrow
↩ \hookleftarrow      ↪ \hookrightarrow		↘ \searrow
↼ \leftharpoonup      ⇀ \rightharpoonup		↙ \swarrow
↽ \leftharpoondown    ⇁ \rightharpoondown	↖ \nwarrow
⇌ \rightleftharpoons  ↝ \leadsto
\end{verbatim}
Openings

\begin{verbatim}[ \lbrack	⌊ \lfloor	⌈ \lceil	⟦ [\![
{ \lbrace	⟨ \langle	⟪ \langle\!\langle
\end{verbatim}
Closings

\begin{verbatim}] \rbrack	⌋ \rfloor	⌉ \rceil	⟧ ]\!]
} \rbrace	⟩ \rangle	⟫ \rangle\!\rangle
\end{verbatim}
Alternate names

\begin{verbatim}≠ \ne or \neq	(same as \not=)
≤ \le		(same as \leq)
≥ \ge		(same as \geq)
{ \{		(same as \lbrace)
} \}		(same as \rbrace)
→ \to		(same as \rightarrow)
← \gets		(same as \leftarrow)
∋ \owns		(same as \ni)
∧ \land		(same as \wedge)
∨ \lor		(same as \vee)
¬ \lnot		(same as \neg)
∣ \vert		(same as |)
∥ \Vert		(same as |\)
\end{verbatim}
Misc

\begin{verbatim}⋮ \vdots
⋯ \cdots
⋱ \ddots
\end{verbatim}
Non-math mode symbols:

Typographic symbols and ligatures

\begin{verbatim}’ '
‘ `
” ''
“ ``
‐ -		(hyphen)
– --		(en dash)
— ---		(em dash)
− $-$		(minus)
′ $'$		(prime)
″ $''$		(double prime)
‴ $'''$		(triple prime)
⁗ $''''$	(quadruple prime)
ff ff
fi fi
fl fl
ffi ffi
ffl ffl
¡ !`
¿ ?`
  ~		(no-break space)
  \thinspace
  $\,$
  \             (space)
œ \oe
Π\OE
æ \ae
Æ \AE
å \aa
Å \AA
ø \o
Ø \O
ł \l
Ł \L
ß \ss
§ \S
¶ \P
† \dag
‡ \ddag
© \copyright
£ \pounds
… \ldots
\end{verbatim}
Combining characters

\begin{verbatim}◌́  \'
◌̀  \`
◌̂  \^
◌̈  \"
◌̃  \~
◌̄  \=
◌̇  \.
◌̆  \u
◌̌  \v
◌̋  \H
◌͡  \t
◌̧  \c
◌̣  \d
◌̱  \b
◌̸  \not
\end{verbatim}
TeX names followed by ! indicate that there is no Unicode character available to encode this exact TeX character (Unicode 3.2 lacks a dot-less j and a non-superscript prime).

\vfill
\hrule
\vspace{1.2mm}
\begin{tiny}
Created by \href{http://maruku.rubyforge.org}{Maruku}  at 01:26 on Tuesday, January 02nd, 2007.
\end{tiny}
\end{document}
